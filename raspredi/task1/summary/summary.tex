\documentclass[12pt, oneside, a4paper]{article}
\usepackage[utf8]{inputenc}
%\usepackage[cp1251]{inputenc} % кодировка
\usepackage[english, russian]{babel} % Русские и английские переносы
\usepackage{graphicx}          % для включения графических изображений
%\usepackage{cite}              % для корректного оформления литературы
%\usepackage{amsmath}                                
%\usepackage{amssymb} 
\usepackage{pavt-ru}
\usepackage{pgfplots}
\pgfplotsset{compat=1.9}
\usepackage{listings}
\usepackage{xcolor}

\definecolor{codegreen}{rgb}{0,0.6,0}
\definecolor{codegray}{rgb}{0.5,0.5,0.5}
\definecolor{codepurple}{rgb}{0.58,0,0.82}
\definecolor{backcolour}{rgb}{0.95,0.95,0.92}

\lstdefinestyle{mystyle}{
    backgroundcolor=\color{backcolour},   
    commentstyle=\color{codegreen},
    keywordstyle=\color{magenta},
    numberstyle=\tiny\color{codegray},
    stringstyle=\color{codepurple},
    basicstyle=\ttfamily\footnotesize,
    breakatwhitespace=false,         
    breaklines=true,                 
    captionpos=b,                    
    keepspaces=true,                 
    numbers=left,                    
    numbersep=5pt,                  
    showspaces=false,                
    showstringspaces=false,
    showtabs=false,                  
    tabsize=2
}

\lstset{style=mystyle}

%Содержимое документа
\begin{document}

\title{Отчет о выполнении первого задания практикума по предмету "Распределенные системы"}
\authors{Р.М.~Куприй, 423 группа}
\organizations{Факультет ВМК МГУ имени М.В.~Ломоносова}

\section{Постановка задания}

Все 16 процессов, находящихся на разных ЭВМ сети, одновременно выдали запрос на вход в критическую секцию. Реализовать программу, использующую древовидный маркерный алгоритм для прохождения всеми процессами критических секций.


Критическая секция:

<проверка наличия файла “critical.txt”>;

if (<файл “critical.txt” существует>) {

<сообщение об ошибке>;

<завершение работы программы>;

} else {

<создание файла “critical.txt”>;

Sleep (<случайное время>);

<уничтожение файла “critical.txt”>;

}


Для передачи маркера использовать средства MPI.


Получить временную оценку работы алгоритма. Оценить сколько времени потребуется, если маркером владеет нулевой процесс. Время старта (время «разгона» после получения доступа к шине для передачи сообщения) равно 100, время передачи байта равно 1 (Ts=100,Tb=1). Процессорные операции, включая чтение из памяти и запись в память, считаются бесконечно быстрыми.

\section{Описание алгоритма}

Алгоритм маркерного дерева \textbf{Raymond}

Для написания программы использовался Древовидный маркерный алгоритм Рэймонда:

Попадание в критическую секцию(далее КС):

Если есть маркер, то процесс выполняет КС;

Если маркера нет, то процесс:

Помещает запрос в очередь запросов;

Посылает сообщение Запрос в направлении владельца маркера и ожидает сообщений;

Поведение при получении сообщений(есть 2 типа сообщений - Запрос и Маркер):

Пришло сообщение Маркер:

Взять 1-ый процесс из очереди и послать маркер автору(может быть, себе);

Поменять значение указателя в сторону маркера на актуальное;

Исключить запрос из очереди;

Если в очереди есть запросы, отправить Запрос в направлении владельца маркера;

Пришло сообщение Запрос:

Поместить запрос в очередь;

Если маркера нет, отправить Запрос в направлении маркера;

Иначе перейти к пункту 1 для Маркер'а

\section{Программная реализация}

Для реализации алгоритма спользовался язык С++ с средствами MPI для параллелизации.

Программа основана на использовании структуры дерева со вспомогательными методами и следующими членами класса: рангом процесса, рангами родителя и потомков, направлением на владельца маркера, флагом наличия маркера и очередью запросов.

\begin{figure}[h]
\begin{lstlisting}[language=C++]
struct tree {
  std::queue<int> request_queue;
  int rank;
  int root;
  int left;
  int right;
  int marker_pointer; // 0 - parent, 1 - left, 2 -right
  bool marker;

  tree() {}

  void critical();

  void receive(message_type message, int sender);
  
  void print();
}
\end{lstlisting}
\end{figure}

Доступ к критической секции реализован в виде метода класса:

\begin{figure}[h]
\begin{lstlisting}[language=C++]
void critical() {
  if (access("critical.txt", F_OK) != -1) {
    std::cerr << "File exist" << std::endl;
    MPI_Finalize();
    exit(1);
  } else {
    fopen("critical.txt", "w");
    std::cout << "CRITICAL; rank: " << rank << std::endl;
    sleep(3);
    remove("critical.txt");
  }
}
\end{lstlisting}
\end{figure}

Основная функция обработки запросов приведена ниже.

\begin{figure}[h]
\begin{lstlisting}[language=C++]
void receive(message_type message, int sender) {
  if (message == MARKER) {

    if (!request_queue.empty()) {
      int requester = request_queue.front();
      request_queue.pop();

      std::cout << "send marker from " << rank << " to " << requester
                << std::endl;

      if (requester == rank) {
        marker = true;
        critical();
      } else if (requester == root) {
        marker = false;
        marker_pointer = 0;
        message_type tmp = MARKER;
        MPI_Send(&tmp, 1, MPI_INT8_T, requester, 0, MPI_COMM_WORLD);
      } else if (requester == left) {
        marker = false;
        marker_pointer = 1;
        message_type tmp = MARKER;
        MPI_Send(&tmp, 1, MPI_INT8_T, requester, 0, MPI_COMM_WORLD);
      } else if (requester == right) {
        marker = false;
        marker_pointer = 2;
        message_type tmp = MARKER;
        MPI_Send(&tmp, 1, MPI_INT8_T, requester, 0, MPI_COMM_WORLD);
      } else {
        std::cerr << "Invalid rank in queue!" << std::endl;
        MPI_Finalize();
        exit(2);
      }

      if (!request_queue.empty()) {
        int receiver = request_queue.front();
        request_queue.pop();
        receive(REQUEST, receiver);
      }
    }
  } else if (message == REQUEST) {
    request_queue.push(sender);
    if (marker) {
      receive(MARKER, rank);
    } else {
      if (marker_pointer == 0) {
        message_type tmp = REQUEST;
        MPI_Send(&tmp, 1, MPI_INT8_T, root, 0, MPI_COMM_WORLD);
      } else if (marker_pointer == 1) {
        message_type tmp = REQUEST;
        MPI_Send(&tmp, 1, MPI_INT8_T, left, 0, MPI_COMM_WORLD);
      } else if (marker_pointer == 2) {
        message_type tmp = REQUEST;
        MPI_Send(&tmp, 1, MPI_INT8_T, right, 0, MPI_COMM_WORLD);
      } else {
        std::cerr << "Invalid marker pointer" << std::endl;
        MPI_Finalize();
        exit(3);
      }
    }
  }
}
\end{lstlisting}
\end{figure}

При посылки сообщения с маркером самому себе - происходит вход в критическую секцию.

\section{Временная оценка}

По условию задачи мы имеем 16 процессов, которые образуют сбалансированное дерево. Для инициализации необходимо 15 запросов. Затем нужно обойти дерево в глубину. Начнем обход с крайней (15) вершины, тогда для этого потребуется 23 операции. В итоге результирующая формула будет выглядеть так: $38*(Ts+1*Tb)$.

\end{document}
